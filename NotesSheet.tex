\documentclass[letter, 12pt]{article}
\usepackage[margin=0.8in,paperwidth=8.5in, paperheight=11in]{geometry}
\usepackage{graphicx,longtable, stmaryrd, ulem, setspace,listings,enumerate,tikz,fancyhdr,multicol, hyperref, calrsfs, float,ifpdf, url, amsmath, amssymb, comment,color,xcolor}
\usepackage[mathscr]{euscript}
\pagestyle{fancy}
\renewcommand{\headrulewidth}{0pt}
\providecommand{\e}[1]{\ensuremath{\times 10^{#1}}}
\newcommand{\red}[1]{\textcolor{red}{#1}}
\newcommand{\blue}[1]{\textcolor{blue}{#1}}
\newcommand{\green}[1]{\textcolor{green}{#1}}
\newcommand{\grey}[1]{\textcolor{gray}{#1}}
\newcommand{\ohm}{$\Omega$}
\DeclareMathOperator{\Error}{Error}
\allowdisplaybreaks
\graphicspath {{figures/}}
\usepackage[utf8]{inputenc}
 \usepackage{listings}
\usepackage{color}
\definecolor{dkgreen}{rgb}{0,0.6,0}
\definecolor{gray}{rgb}{0.5,0.5,0.5}
\definecolor{mauve}{rgb}{0.58,0,0.82}
\lstset{frame=tb,
  language=R,
  aboveskip=3mm,
  belowskip=3mm,
  showstringspaces=false,
  columns=flexible,
  basicstyle={\small\ttfamily},
  numbers=none,
  numberstyle=\tiny\color{gray},
  keywordstyle=\color{blue},
  commentstyle=\color{dkgreen},
  stringstyle=\color{mauve},
  breaklines=true,
  breakatwhitespace=true,
  tabsize=3
}
\begin{document}
\begin{center}
STAT 4033
\end{center}
Exam 1 Notes \hfill Name: \uline{Cooper Morris}
\begin{multicols}{2}
\textbf{\uline{C1S1:}}\\
\textbf{Population:} the entire set of all potential measurement\\
\textbf{Sample:} any subset of a population\\
\textbf{Simple Random Sample:} A sample of size \textit{n} taken in such a way that any group of size \textit{n} has the same chance of being selected\\
\textbf{Sampling Variability:} different samples from the same population can lead to differences\\
\textbf{Stratified Random Sampling:} the population is broken into groups based off a characteristic. Then a SRS is taken from each group\\
\textbf{Cluster Sampling:} target population has many groups, groups are selected by SRS of the groups. All elements of each group are selected\\
\textbf{Systematic Sample:} A listing is generated over time, every \textit{k}\textsuperscript{th} member is included in the sample.\\
\textbf{Tangible Population:} A population composed of members/individuals that exist.\\
\textbf{Conceptual Population:} A population composed of all values that can potentially be observed. They do not necessarily exist at any point in time.\\
\textbf{Observations:} The measurement, or set of measurements recorded from any individual in a sample.\\
\textbf{Variables:} The characteristics being observed from individuals.\\
\textbf{Quanitative Variables:} Possible values that represent \textit{quantiles of something.} Numbers of things.\\
\textbf{Ratio Variables:} Inherent zero value and ratios between values make sense.
\textbf{Interval Variables:} No meaningful ratios and arbitrary zero \\
\textbf{Qualitative:} A variable that takes a category of possible values.\\
\textbf{Nominal:} Ordering of categories makes sense.\\
\textbf{Ordinal:} No inherent ranking in categories.\\
\textbf{Observational Study:} Observe a sample from a population with minimal interaction.\\
\textbf{Experimental Study:} A study performed where the environment of subjects is strictly controlled.\\
\textbf{Response Variable(s):} The variable(s) of interest in a study.\\
\textbf{Explanatory Variable:} Variables to explain changes in the response variable.\\
\textbf{Confounding Variable(s):} Variables unaccounted for i a study that may explain changes in the response variable.

\vfill
\columnbreak
\vspace*{\fill}
\end{multicols}
\end{document}