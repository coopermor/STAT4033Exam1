\documentclass[letter, 12pt]{article}
\usepackage[margin=0.6in,paperwidth=8.5in, paperheight=11in]{geometry}
\usepackage{graphicx,longtable, stmaryrd, ulem, setspace,listings,enumerate,tikz,fancyhdr,multicol, hyperref, calrsfs, float,ifpdf, url, amsmath, amssymb, comment,color,xcolor,dsfont}
\usepackage[mathscr]{euscript}
\pagestyle{fancy}

\renewcommand{\headrulewidth}{0pt}
\providecommand{\e}[1]{\ensuremath{\times 10^{#1}}}
\newcommand{\red}[1]{\textcolor{red}{#1}}
\newcommand{\blue}[1]{\textcolor{blue}{#1}}
\newcommand{\green}[1]{\textcolor{green}{#1}}
\newcommand{\grey}[1]{\textcolor{gray}{#1}}
\newcommand{\ohm}{$\Omega$}
\DeclareMathOperator{\Error}{Error}
\allowdisplaybreaks
\graphicspath {{figures/}}
\usepackage[utf8]{inputenc}
 \usepackage{listings}
\usepackage{color}
\definecolor{dkgreen}{rgb}{0,0.6,0}
\definecolor{gray}{rgb}{0.5,0.5,0.5}
\definecolor{mauve}{rgb}{0.58,0,0.82}
\lstset{frame=tb,
  language=R,
  aboveskip=3mm,
  belowskip=3mm,
  showstringspaces=false,
  columns=flexible,
  basicstyle={\small\ttfamily},
  numbers=none,
  numberstyle=\tiny\color{gray},
  keywordstyle=\color{blue},
  commentstyle=\color{dkgreen},
  stringstyle=\color{mauve},
  breaklines=true,
  breakatwhitespace=true,
  tabsize=3
}
\begin{document}
\begin{center}
STAT 4033
\end{center}
Exam 3 Notes \hfill Name: \uline{Cooper Morris}
\begin{multicols}{2}
\textbf{\uline{C4S6:}}\\
\textbf{\uline{The Lognormal Distribution}}\\
\(Y \sim Lognorm(\mu, \sigma)\)\\
\(E[Y] = e^{\mu+\sigma^2/2}\)\\
\(Var(Y) = \sigma_y^2 = e^{2\mu} \cdot (e^{2\sigma^2} - e^{\sigma^2})\)\\
\(P(a\leq Y \leq b) = P(\frac{\ln a -\mu}{\sigma} \leq Z \leq \frac{\ln b -\mu}{\sigma})\)\\
\textbf{\uline{C4S7:}}\\
\textbf{\uline{The Exponential Distribution}}\\
\(Y \sim Exp(\lambda)\)\\
\(E[Y] = \frac{1}{\lambda}\)\\
\(Var(Y) = \frac{1}{\lambda^2}\)\\
\(P(X>x) = e^{-\lambda\cdot x}\)\\
\(P(X\leq x) = 1-e^{-\lambda\cdot x}\)\\
\textbf{\uline{C4S11:}}\\
\textbf{\uline{The Central Limit Theorem}}\\
\(\bar{X} \dot{\sim} N(\mu, \frac{\sigma}{\sqrt{n}})\)\\
\(z = \frac{x-\mu}{\frac{\sigma}{\sqrt{n}}}\)\\
Let \(X\sim Bin(n,p)\) given sufficient sample size \(X\dot{\sim} N(np, np(1-p))\)\\
Sufficient sample size if \(n\cdot\hat{p}\) and \(n\cdot(1-\hat{p})\) are both greater than 10\\
\(\hat{p}\dot{\sim} N(p, \frac{p(1-p)}{n}\)\\
Let \(X \sim Poi(\lambda)\) if \(\lambda\) is large enough we can use \(X\dot{\sim} N(\lambda, \lambda)\)\\
\(z = \frac{x-\lambda}{\sqrt{\lambda}}\)\\
Sample size is sufficient if \(\lambda > 10\)\\
\textbf{\uline{C5S1:}}\\
\textbf{\uline{Large Sample Confidence Intervals}}\\
\(\bar{x} \pm z_{\alpha/2}\cdot\frac{\sigma}{\sqrt{n}}\)\\
Sufficient sample size: \(n \geq (\frac{z_{\alpha/2}\cdot\sigma}{E})^2\)\\
Always round sample size up to next integer\\
Get z value from last row of Table A3 where \(\nu = \infty\)\\
\begin{tabular}{c|cccc}
Confidence Level & 90\%  & 95\%  & 98\%  & 99\%  \\ \hline
\(\alpha\)           & 0.10  & 0.05  & 0.02  & 0.01  \\ \hline
\(z_{\alpha/2}\)  & 1.645 & 1.960 & 2.326 & 2.576
\end{tabular}
\textbf{\uline{C5S2:}}\\
\textbf{\uline{Confidence Intervals for Proportions}}\\
\(\tilde{n} = n+4\) and \(\tilde{p} = \frac{x+2}{n+4}\)\\
\(\tilde{p} \pm z_{\alpha}\cdot\sqrt{\frac{\tilde{p}(1-\tilde{p})}{\tilde{n}}}\)\\
\textbf{\uline{C5S3:}}\\
\textbf{\uline{Small Sample Confidence Intervals}}\\
\(\bar{x} \pm t_{\alpha/2, \nu}\cdot\frac{s}{\sqrt{n}}\)\\
Sufficient sample size: \(n \geq (\frac{t_{\alpha/2, \nu}\cdot s}{E})^2\)\\

\textbf{\uline{C5S6:}}\\
\textbf{\uline{Small Sample Confidence Intervals for Difference Between Two Means}}\\
If \(\sigma_1 = \sigma_2\)\\
\(s_p = \sqrt{\frac{(n_1)s^2_1+(n_2-1)s^2_2}{n_1+n_2-2}}\)\\
\(\nu = n_1+n_2-2\)\\
\((\bar{x_1}-\bar{x_2})\pm t_{\alpha/2,v}\cdot s_p\cdot\sqrt{\frac{1}{n_1}+\frac{1}{n_2}}\)\\
If \(\sigma_1 \neq \sigma_2\)\\
\[\nu = \frac{(\frac{s^2_1}{n^1}+\frac{s^2_2}{n_2})^2}{\frac{(\frac{s^2_1}{n_1})^2}{n_1-1}+\frac{(\frac{s^2_2}{n_2})^2}{n_2-1}}\]\\
\((\bar{x_1}-\bar{x_2})\pm t_{\alpha/2,v}\cdot\sqrt{\frac{s_1^2}{n_1}+\frac{s^2_2}{n_2}}\)\\
\textbf{\uline{C5S9:}}\\
\textbf{\uline{Confidence Intervals}}\\
Prediction Intervals:\\
\(\bar{x}\pm t_{\alpha/2, \nu}\cdot s \cdot \sqrt{1+\frac{1}{n}}\)\\
Tolerance Intervals:
\(\bar{x} \pm k_{n, \alpha, \gamma} \cdot s\)\\
Use Table A4.\\
\(\gamma\) contains x\% of the sample\\
\(\alpha\) is with x\% confidence\\
%%%%%%%%%%%%%%%%%%%%%%%%%%%%%%%%%%%%%%%%%%%%%%%%%%%%%%%%%%%%
\end{multicols}

\end{document}